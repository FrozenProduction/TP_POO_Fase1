\section{Conclusão}

Neste capítulo, são apresentadas as conclusões retiradas do trabalho desenvolvido nesta primeira fase, bem como as linhas de orientação para o trabalho futuro a desenvolver na Fase 2.

\subsection{Conclusões da Fase 1}
A Fase 1 do projeto permitiu estabelecer as fundações de um sistema robusto para a gestão de um jardim zoológico. Todos os objetivos propostos para esta fase foram alcançados: a arquitetura de \textit{software} (DLL + Aplicação de Consola) foi implementada com sucesso, cumpre os requisitos de modularidade e reutilização.

As hierarquias de classes desenhadas (visíveis na Figura \ref{fig:diagrama_classes}) demonstraram ser eficazes na modelação do problema, e a aplicação dos pilares da POO (Abstração, Herança, Encapsulamento e Polimorfismo) permitiu criar um código limpo, extensível e de fácil manutenção, cobrindo todos os requisitos funcionais essenciais do enunciado.

\subsection{Trabalho Futuro (Fase 2)}
A Fase 2 incidirá em completar a solução, adicionando funcionalidades avançadas e garantindo a qualidade do \textit{software}, conforme os critérios de avaliação. As principais tarefas a desenvolver serão:
\begin{itemize}
    \item \textbf{Persistência de dados:} Implementar a capacidade de guardar e carregar o estado do zoológico (listas de animais e tarefas) em ficheiros.
    \item \textbf{Tratamento de Exceções:} Implementar exceções personalizadas (ex: \texttt{AnimalNaoEncontradoException}) para um tratamento de erros robusto.
    \item \textbf{Testes Unitários:} Desenvolver um projeto de testes unitários para validar a lógica de negócio, visando uma cobertura de código significativa.
    \item \textbf{Utilização de LINQ:} Refatorar o código de procura e gestão de listas para utilizar expressões LINQ, como pedido.
    \item \textbf{Aplicação Demonstradora:} Melhorar a aplicação de consola para se tornar uma interface de utilizador mais interativa, que demonstre todos os serviços implementados.
\end{itemize}