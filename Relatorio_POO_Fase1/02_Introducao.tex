\section{Introdução}

O presente capítulo expõe a contextualização do projeto. Inicia-se com a motivação e o enquadramento do trabalho, seguindo-se a definição dos objetivos traçados para esta primeira fase. Por fim, é detalhada a metodologia de trabalho adotada e a estrutura deste documento.

\subsection{Motivação e Enquadramento}

A gestão de um jardim zoológico moderno é uma operação logística complexa que envolve o cuidado de múltiplas espécies (animais), cada uma com necessidades distintas, e a coordenação de diversas tarefas diárias, desde a alimentação e limpeza até à assistência veterinária e realização de espetáculos. A ausência de um sistema de gestão centralizado e extensível pode levar a ineficiências e erros.

O paradigma da Programação Orientada a Objetos (POO) apresenta-se como a metodologia ideal para modelar este problema, permitindo criar representações digitais de entidades do mundo real (como \texttt{Animal} ou \texttt{Tarefa}) e gerir as suas interações.

Este projeto é desenvolvido no contexto da Unidade Curricular de Programação Orientada a Objetos, da Licenciatura em Engenharia de Sistemas Informáticos do Instituto Politécnico do Cávado e do Ave.

\subsection{Objetivos}

Para a conclusão da Fase 1 deste projeto, definiram-se os seguintes objetivos:
\begin{itemize}
    \item Analisar os requisitos do problema de "Gestão de Jardim Zoológico".
    \item Desenhar uma arquitetura de \textit{software} modular, separando a lógica de negócio (\textit{Core}) da aplicação de teste (Consola).
    \item Identificar e modelar as classes e hierarquias essenciais através de um diagrama UML.
    \item Implementar a estrutura de classes base, aplicando corretamente os pilares da POO (Abstração, Encapsulamento, Herança e Polimorfismo).
    \item Utilizar estruturas de dados adequadas para a gestão das coleções de animais e tarefas.
\end{itemize}

\subsection{Metodologia de Trabalho}

A metodologia adotada foi o desenvolvimento orientado a objetos. O processo iniciou-se pela análise dos requisitos (palavras-chave do enunciado), seguindo-se a fase de desenho e modelação, onde as entidades e as suas relações foram definidas num diagrama de classes. Posteriormente, procedeu-se à implementação em C\#, criando uma biblioteca de classes reutilizável e uma aplicação de consola para testes, garantindo que a "implementação essencial" estivesse funcional.

\subsection{Estrutura do Documento}

Este documento está organizado em cinco capítulos. O Capítulo 1 (o atual) introduz o projeto. O Capítulo 2 aborda o enquadramento teórico, focando-se nos pilares da POO. O Capítulo 3 detalha todo o trabalho desenvolvido, desde a arquitetura da solução até à implementação de cada pilar. Por fim, o Capítulo 4 apresenta a conclusão desta fase e aponta o trabalho futuro a desenvolver na Fase 2.