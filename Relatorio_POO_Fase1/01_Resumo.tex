\section*{Resumo}
\addcontentsline{toc}{section}{Resumo} % Adiciona "Resumo" ao Índice

O presente trabalho, desenvolvido no âmbito da Unidade Curricular de Programação Orientada a Objetos, foca-se na aplicação dos conceitos do paradigma orientado a objetos para resolver um problema de gestão de tarefas de um jardim zoológico. 

Este relatório documenta a Fase 1 do projeto, que compreende a análise de requisitos, o desenho da arquitetura e a implementação essencial das classes. A solução foi estruturada numa biblioteca de classes (\textit{DLL}) para promover a modularidade, e uma aplicação de consola para demonstração.

O desenho da solução assenta em hierarquias de classes abstratas (como \texttt{Animal} e \texttt{Tarefa}) que utilizam os pilares da abstração, herança e polimorfismo para gerir as diferentes entidades e operações do parque, como a alimentação, assistência veterinária e gestão de bilhetes. O resultado desta fase é um núcleo de \textit{software} funcional que cumpre todos os requisitos essenciais definidos.