% --- PREÂMBULO ---
\documentclass[a4paper, 12pt]{article}

% --- Pacotes Essenciais ---
\usepackage[utf8]{inputenc}    % Acentuação correta
\usepackage[T1]{fontenc}       % Fontes modernas
\usepackage[portuguese]{babel} % Define o idioma para Português (PT-PT)
\usepackage[left=2.5cm, right=2.5cm, top=2.5cm, bottom=2.5cm]{geometry} % Margens

% --- Pacotes para Gráficos e Imagens ---
\usepackage{graphicx}          % Para incluir imagens
\usepackage{float}             % Para controlar a posição das figuras (ex: [H])
\graphicspath{ {imagens/} }    % Define a pasta padrão para as imagens

% --- Cabeçalhos e Rodapés ---
\usepackage{fancyhdr}
\pagestyle{fancy}
\fancyhf{} % Limpa todos os campos
\fancyfoot[C]{\thepage} % Número da página no centro do rodapé
\renewcommand{\headrulewidth}{0.4pt} % Linha no cabeçalho
\fancyhead[L]{Programação Orientada a Objetos}
\fancyhead[R]{Fase 1}

% --- Outros ---
\usepackage{hyperref}          % Links clicáveis (para o índice e URLs)
\usepackage{parskip}           % Espaço entre parágrafos em vez de indentação
\usepackage{tocbibind}         % Adiciona Resumo e Índice ao Índice

% --- FIM DO PREÂMBULO ---

% --- DOCUMENTO ---
\begin{document}

% 1. Página de Capa
\begin{titlepage}
    \centering
    
    \includegraphics[width=0.4\textwidth]{ipca_logo.png} % Descomenta se tiveres o logo
    
    \vspace{1.5cm}
    
    {\Huge \textbf{Instituto Politécnico do Cávado e do Ave}} \\
    \vspace{1cm} % <-- ESPAÇO ADICIONADO AQUI
    {\Large Escola Superior de Tecnologia} \\
    
    \vspace{1cm}
    
    {\Large Licenciatura em Engenharia de Sistemas Informáticos} \\
    
    \vspace{1.5cm}
    
    {\huge \textbf{Gestão de Jardim Zoológico}} \\
    \vspace{0.5cm}
    {\Large Relatório de Projeto - Fase 1} \\
    
    \vfill % Empurra o texto seguinte para o fundo
    
    \begin{flushleft}
    \Large \textbf{Unidade Curricular:} Programação Orientada a Objetos \\
    \Large \textbf{Docentes:} Prof. Luís G. Ferreira \\
    
    \vspace{0.5cm}
    \Large \textbf{Autor:} \\
    Diogo Pereira \\
    31513 \\
    a31513@alunos.ipca.pt
    \end{flushleft}
    
    \vspace{1cm}
    
    {\large Barcelos, 12 de novembro de 2025} % Altera para a data correta
    
\end{titlepage}

% 2. Resumo
\newpage
\section*{Resumo}
\addcontentsline{toc}{section}{Resumo} % Adiciona "Resumo" ao Índice

O presente trabalho, desenvolvido no âmbito da Unidade Curricular de Programação Orientada a Objetos, foca-se na aplicação dos conceitos do paradigma orientado a objetos para resolver um problema de gestão de tarefas de um jardim zoológico. 

Este relatório documenta a Fase 1 do projeto, que compreende a análise de requisitos, o desenho da arquitetura e a implementação essencial das classes. A solução foi estruturada numa biblioteca de classes (\textit{DLL}) para promover a modularidade, e uma aplicação de consola para demonstração.

O desenho da solução assenta em hierarquias de classes abstratas (como \texttt{Animal} e \texttt{Tarefa}) que utilizam os pilares da abstração, herança e polimorfismo para gerir as diferentes entidades e operações do parque, como a alimentação, assistência veterinária e gestão de bilhetes. O resultado desta fase é um núcleo de \textit{software} funcional que cumpre todos os requisitos essenciais definidos.

% 3. Índice
\newpage
\tableofcontents % Gera o Índice automaticamente
\thispagestyle{empty} % Remove cabeçalho/rodapé da página do índice

% 4. Corpo do Relatório
\newpage
\section{Introdução}

O presente capítulo expõe a contextualização do projeto. Inicia-se com a motivação e o enquadramento do trabalho, seguindo-se a definição dos objetivos traçados para esta primeira fase. Por fim, é detalhada a metodologia de trabalho adotada e a estrutura deste documento.

\subsection{Motivação e Enquadramento}

A gestão de um jardim zoológico moderno é uma operação logística complexa que envolve o cuidado de múltiplas espécies (animais), cada uma com necessidades distintas, e a coordenação de diversas tarefas diárias, desde a alimentação e limpeza até à assistência veterinária e realização de espetáculos. A ausência de um sistema de gestão centralizado e extensível pode levar a ineficiências e erros.

O paradigma da Programação Orientada a Objetos (POO) apresenta-se como a metodologia ideal para modelar este problema, permitindo criar representações digitais de entidades do mundo real (como \texttt{Animal} ou \texttt{Tarefa}) e gerir as suas interações.

Este projeto é desenvolvido no contexto da Unidade Curricular de Programação Orientada a Objetos, da Licenciatura em Engenharia de Sistemas Informáticos do Instituto Politécnico do Cávado e do Ave.

\subsection{Objetivos}

Para a conclusão da Fase 1 deste projeto, definiram-se os seguintes objetivos:
\begin{itemize}
    \item Analisar os requisitos do problema de "Gestão de Jardim Zoológico".
    \item Desenhar uma arquitetura de \textit{software} modular, separando a lógica de negócio (\textit{Core}) da aplicação de teste (Consola).
    \item Identificar e modelar as classes e hierarquias essenciais através de um diagrama UML.
    \item Implementar a estrutura de classes base, aplicando corretamente os pilares da POO (Abstração, Encapsulamento, Herança e Polimorfismo).
    \item Utilizar estruturas de dados adequadas para a gestão das coleções de animais e tarefas.
\end{itemize}

\subsection{Metodologia de Trabalho}

A metodologia adotada foi o desenvolvimento orientado a objetos. O processo iniciou-se pela análise dos requisitos (palavras-chave do enunciado), seguindo-se a fase de desenho e modelação, onde as entidades e as suas relações foram definidas num diagrama de classes. Posteriormente, procedeu-se à implementação em C\#, criando uma biblioteca de classes reutilizável e uma aplicação de consola para testes, garantindo que a "implementação essencial" estivesse funcional.

\subsection{Estrutura do Documento}

Este documento está organizado em cinco capítulos. O Capítulo 1 (o atual) introduz o projeto. O Capítulo 2 aborda o enquadramento teórico, focando-se nos pilares da POO. O Capítulo 3 detalha todo o trabalho desenvolvido, desde a arquitetura da solução até à implementação de cada pilar. Por fim, o Capítulo 4 apresenta a conclusão desta fase e aponta o trabalho futuro a desenvolver na Fase 2.

\newpage
\section{Enquadramento Teórico}

Neste capítulo são revistos os fundamentos teóricos do Paradigma de Programação Orientada a Objetos (POO), que constituem a base metodológica sobre a qual este projeto foi desenvolvido. A correta aplicação destes conceitos é um dos objetivos centrais do trabalho.

\subsection{Paradigma de Programação Orientada a Objetos}

A POO é um paradigma de programação baseado no conceito de "objetos", que podem conter dados (atributos) e código (métodos). O objetivo é modelar o \textit{software} de uma forma mais próxima da realidade, permitindo uma maior reutilização de código, flexibilidade e facilidade de manutenção. Este paradigma assenta em quatro pilares fundamentais, descritos de seguida.

\subsection{Pilares da POO}

\subsubsection{Abstração}
A abstração consiste em focar-se nos aspetos essenciais de uma entidade, ignorando os detalhes supérfluos. Em C\#, isto é frequentemente implementado através de classes abstratas (\textit{abstract class}) ou interfaces. Uma classe abstrata define um "contrato" do que uma classe deve ser, mas não pode ser instanciada diretamente, obrigando outras classes a herdar dela.

\subsubsection{Encapsulamento}
O encapsulamento é o mecanismo que restringe o acesso direto aos dados de um objeto, protegendo-os de modificações indevidas. Os dados (atributos) são definidos como privados (\textit{private}) e o acesso a eles é controlado através de métodos públicos (\textit{public}), como propriedades (\textit{getters} e \textit{setters}).

\subsubsection{Herança}
A herança permite que uma classe (subclasse ou classe filha) adquira os atributos e métodos de outra classe (superclasse ou classe pai). Isto promove a reutilização de código e a criação de hierarquias lógicas. A classe filha pode especializar a classe pai, adicionando novos métodos ou modificando (através de \textit{override}) os existentes.

\subsubsection{Polimorfismo}
O polimorfismo (do grego, "muitas formas") é a capacidade de um objeto assumir diferentes formas. Em termos práticos, permite que uma variável de um tipo base (superclasse) possa referenciar um objeto de um tipo derivado (subclasse). Isto é fundamental para tratar coleções de objetos diferentes de forma homogénea, como uma lista de \texttt{Tarefa} que contém objetos \texttt{Alimentacao} e \texttt{LimpezaJaula}.

\newpage
\section{Trabalho Desenvolvido}

Este capítulo detalha a solução desenhada e implementada para a Fase 1, justificando as decisões tomadas com base nos requisitos do enunciado e nos conceitos teóricos do capítulo anterior.

\subsection{Análise e Especificação}

A primeira etapa consistiu na análise do problema e no desenho da arquitetura geral da solução.

\subsubsection{Requisitos}
Com base no tema "Gestão de jardim zoológico", foram identificados os seguintes requisitos essenciais (palavras-chave) para a Fase 1:
\begin{itemize}
    \item Gestão de **animais** e das suas **informações**.
    \item Gestão de um **calendário** de tarefas.
    \item Definição de tarefas específicas: **alimentação** (com **tipos de comida**), **assistência veterinária**, **limpeza de jaulas** e **espetáculos**.
    \item Gestão da venda de **bilhetes**.
\end{itemize}

\subsubsection{Arquitetura da Solução}
Para cumprir o requisito de "Reutilização de código através de bibliotecas (DLL)", a solução foi dividida em dois projetos:
\begin{itemize}
    \item \textbf{Zoologico.Core}: Um projeto do tipo "Biblioteca de Classes" (.dll) que contém toda a lógica de negócio, incluindo as classes de animais, tarefas e a classe principal de gestão.
    \item \textbf{Zoologico.App}: Um projeto "Aplicação de Consola" (.exe) que serve como "Aplicação demonstradora". Este projeto adiciona uma referência ao \texttt{Zoologico.Core} para poder instanciar e testar as suas classes.
\end{itemize}
Esta separação garante um baixo acoplamento e uma clara separação de responsabilidades.

\subsubsection{Modelação (Diagrama de Classes)}
O desenho da solução foi modelado num diagrama de classes UML, conforme exigido. A Figura \ref{fig:diagrama_classes} ilustra as principais classes e as suas relações de herança e associação.

\begin{figure}[H]
    \centering
    \includegraphics[width=1\textwidth]{diagrama_classes.png}
    \caption{Diagrama de Classes UML da solução.}
    \label{fig:diagrama_classes}
    \small{\textit{Fonte: Ficheiro \texttt{DiagramaPOO.cd} do projeto.}}
\end{figure}

\subsection{Implementação e Decisões de Desenho}

A implementação focou-se na aplicação correta dos pilares da POO para garantir uma solução robusta e extensível.

\subsubsection{Abstração (Animal e Tarefa)}
Decidiu-se que nem \texttt{Animal} nem \texttt{Tarefa} deveriam ser instanciados diretamente. Definiu-se a classe \texttt{Animal} como \textit{abstract}, pois um "animal" genérico não existe; o que existe são mamíferos, aves, etc. Esta classe define métodos abstratos como \texttt{GetInformacao()}, forçando as classes filhas a implementá-los.

O mesmo princípio foi aplicado à classe \texttt{Tarefa}. Ela define o que é comum a todas as tarefas (uma data e um estado de conclusão), mas a lógica de \texttt{Executar()} é específica de cada subtipo.

\subsubsection{Herança (Animais e Tipos de Tarefa)}
A herança foi usada para criar especializações. As classes \texttt{Mamifero} e \texttt{Ave} herdam de \texttt{Animal}, reutilizando as propriedades \texttt{Nome} e \texttt{Idade}, mas implementando de forma única o método \texttt{GetInformacao()}.

De igual modo, as classes \texttt{Alimentacao}, \texttt{LimpezaJaula}, \texttt{AssistenciaVeterinaria} e \texttt{Espetaculo} herdam de \texttt{Tarefa}, partilhando o construtor base e as suas propriedades.

\subsubsection{Polimorfismo (Gestão de Tarefas)}
O pilar do polimorfismo é central na gestão do zoológico. A classe \texttt{Zoologico} gere um \texttt{List<Tarefa>} (o "calendário"). No método \texttt{ExecutarTarefasPendentes()}, o sistema itera sobre esta lista e invoca o método \texttt{t.Executar()}. Graças ao polimorfismo, o sistema executa a versão correta do método (\textit{override}) para cada objeto, seja ele \texttt{Alimentacao} ou \texttt{LimpezaJaula}, sem necessidade de verificar o seu tipo específico.

\subsubsection{Encapsulamento (Proteção de Dados)}
O encapsulamento foi aplicado para proteger o estado interno dos objetos. Na classe \texttt{Animal}, os atributos \texttt{nome} e \texttt{dataNascimento} são \textit{private}. O acesso é controlado por propriedades \textit{public}, como \texttt{Nome} (que permite leitura e escrita) e \texttt{Idade} (que permite apenas leitura e calcula o valor).

\subsubsection{Métodos de Classe (Static)}
Para o requisito "bilhetes", decidiu-se que o número total de bilhetes vendidos é uma informação global do zoológico, e não de uma instância particular. Para tal, usou-se um atributo \textit{private static int totalBilhetesVendidos} na classe \texttt{Zoologico}. Os métodos \texttt{VenderBilhete()} e \texttt{GetTotalBilhetesVendidos()} foram implementados como \textit{public static}, permitindo a sua invocação diretamente a partir da classe (\texttt{Zoo.VenderBilhete()}).

\newpage
\section{Conclusão}

Neste capítulo, são apresentadas as conclusões retiradas do trabalho desenvolvido nesta primeira fase, bem como as linhas de orientação para o trabalho futuro a desenvolver na Fase 2.

\subsection{Conclusões da Fase 1}
A Fase 1 do projeto permitiu estabelecer as fundações de um sistema robusto para a gestão de um jardim zoológico. Todos os objetivos propostos para esta fase foram alcançados: a arquitetura de \textit{software} (DLL + Aplicação de Consola) foi implementada com sucesso, cumpre os requisitos de modularidade e reutilização.

As hierarquias de classes desenhadas (visíveis na Figura \ref{fig:diagrama_classes}) demonstraram ser eficazes na modelação do problema, e a aplicação dos pilares da POO (Abstração, Herança, Encapsulamento e Polimorfismo) permitiu criar um código limpo, extensível e de fácil manutenção, cobrindo todos os requisitos funcionais essenciais do enunciado.

\subsection{Trabalho Futuro (Fase 2)}
A Fase 2 incidirá em completar a solução, adicionando funcionalidades avançadas e garantindo a qualidade do \textit{software}, conforme os critérios de avaliação. As principais tarefas a desenvolver serão:
\begin{itemize}
    \item \textbf{Persistência de dados:} Implementar a capacidade de guardar e carregar o estado do zoológico (listas de animais e tarefas) em ficheiros.
    \item \textbf{Tratamento de Exceções:} Implementar exceções personalizadas (ex: \texttt{AnimalNaoEncontradoException}) para um tratamento de erros robusto.
    \item \textbf{Testes Unitários:} Desenvolver um projeto de testes unitários para validar a lógica de negócio, visando uma cobertura de código significativa.
    \item \textbf{Utilização de LINQ:} Refatorar o código de procura e gestão de listas para utilizar expressões LINQ, como pedido.
    \item \textbf{Aplicação Demonstradora:} Melhorar a aplicação de consola para se tornar uma interface de utilizador mais interativa, que demonstre todos os serviços implementados.
\end{itemize} %

\newpage
\input{06_GitHub.tex} %

\end{document}